\documentclass{beamer}
\DeclareMathOperator*{\argmax}{arg\,max}

\usecolortheme{whale}
\useinnertheme[shadow]{rounded}
\usenavigationsymbolstemplate{}
\usepackage[subsection=false,headline=empty,footline=outlineauthortitle]{beamerouterthememiniframesbottom}
\usepackage{tikz}
\usetikzlibrary{backgrounds,fit,shapes.misc}

\usepackage{enumitem}

\setitemize{label=\usebeamerfont*{itemize item}%
  \usebeamercolor[fg]{itemize item}
  \usebeamertemplate{itemize item}}

% Set up citation style
\usepackage{natbib}
\usepackage{bibentry}
\bibpunct{(}{)}{;}{a}{,}{,}
\newcommand{\newcite}[1]{\citet{#1}}
\renewcommand{\cite}[1]{\citep{#1}}

\setbeamercovered{transparent}

\newcommand{\homedir}{\raise.17ex\hbox{$\scriptstyle\sim$}}


\title[LING 402]{Tools and Techniques for \\ Speech and Language Processing}
\author[Week 07 of 16 --- Day 14 of 29]{}
\institute[shortinst]{University of Illinois at Urbana-Champaign}

\date{Week 07 of 16 --- Day 14 of 29}

\begin{document}

% Specify that no bibliography should be printed
\bibliographystyle{plainnat}

\frame{\titlepage}


\frame {
\frametitle{Quiz (\homedir10 minutes)}

}

\frame{
\frametitle{hw08 housekeeping}

NB that some lines of code in the scaffolding will need to be replaced:

\begin{itemize}
\item \texttt{pass}
\item \texttt{if True == False}
\item \texttt{while True == False}
\end{itemize}

When you write your replacement, you should only need a single line of code.
}

\frame{
\frametitle{hw08 concepts}
Brief overview of \textbf{classes} and their \textbf{instantiation} as \textbf{objects}
}

\frame {
\frametitle{Play time}

Implement a simple card game in which two players draw a card from a deck and the highest card wins.
\begin{itemize}
\item Implement a \textbf{Card} class which has \textbf{value} and \textbf{suit} attributes.
\item Implement a \textbf{ShuffledDeck} class which instantiates 52 Card of unique suit and value combinations.
\item The ShuffledDeck class must also have a \texttt{draw} function, which returns the top Card.
\item Print out the value and suit of each player's drawn Card with the value as a string (\texttt{Q, J, 10, 9,} etc.) and the suit represented by the \textbf{card suit unicode codepoints}.
\item Print which player wins else print that it was a draw (\textit{how can you compare the card values?}).
\end{itemize}

}

\end{document}
