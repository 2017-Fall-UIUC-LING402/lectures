\documentclass{beamer}
\DeclareMathOperator*{\argmax}{arg\,max}

\usecolortheme{whale}
\useinnertheme[shadow]{rounded}
\usenavigationsymbolstemplate{}
\usepackage[subsection=false,headline=empty,footline=outlineauthortitle]{beamerouterthememiniframesbottom}
\usepackage{tikz}
\usetikzlibrary{backgrounds,fit,shapes.misc}

\usepackage{enumitem}

\setitemize{label=\usebeamerfont*{itemize item}%
  \usebeamercolor[fg]{itemize item}
  \usebeamertemplate{itemize item}}

% Set up citation style
\usepackage{natbib}
\usepackage{bibentry}
\bibpunct{(}{)}{;}{a}{,}{,}
\newcommand{\newcite}[1]{\citet{#1}}
\renewcommand{\cite}[1]{\citep{#1}}

\setbeamercovered{transparent}

\newcommand{\homedir}{\raise.17ex\hbox{$\scriptstyle\sim$}}


\title[LING 402]{Tools and Techniques for \\ Speech and Language Processing}
\author[Week 04 of 16 --- Day 07 of 29]{}
\institute[shortinst]{University of Illinois at Urbana-Champaign}

\date{Week 04 of 16 --- Day 07 of 29}

\begin{document}

% Specify that no bibliography should be printed
\bibliographystyle{plainnat}

\frame{\titlepage}


\frame {
\frametitle{Quiz (\homedir10 minutes)}

}

\frame {
\frametitle{Go over HW04 (\homedir10 minutes)}

}


\begin{frame}[fragile]
\frametitle{Introduce hw05 (\homedir10 minutes)}

\begin{verbatim}
#!/usr/bin/python3

print("Hello, world")
\end{verbatim}

\end{frame}


\frame {
\frametitle{Play time }

	Write a game that prompts the user to guess a number chosen a random from between 0 and 1000.
	
	\begin{itemize}
	
	\item The game should be a bash script.
	
	\item The user should have a limited number of guesses (\homedir10-20).
	
	\item The output of stdout should specify whether the user's guess is too high, too low, or correct.
	
	\item It would also be good to check that the user has actually input a number as a guess.
	
	\end{itemize}
	
	\ \\
	
	If you have time, rewrite the script so that the max number of guesses and upper bound of the number to be guessed 
	can be passed to the script as positional parameters.

}

\end{document}
