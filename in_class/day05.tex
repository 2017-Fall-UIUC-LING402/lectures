\documentclass{beamer}
\input{preamble.tex}


\title[LING 402]{Tools and Techniques for \\ Speech and Language Processing}
\author[Week 03 of 16 --- Day 05 of 29]{}
\institute[shortinst]{University of Illinois at Urbana-Champaign}

\date{Week 03 of 16 --- Day 05 of 29}

\begin{document}

% Specify that no bibliography should be printed
\bibliographystyle{plainnat}

\frame{\titlepage}


\frame {
\frametitle{Quiz (\homedir15 minutes)}

}

\frame{
\frametitle{Practice time}

	The following exercise expects you to:
	\begin{itemize}
		\item Write a bash script that reads from stdin (w/o \texttt{read} command)
		\item Make use of bash variables and expansions
		\item Get the output of a single python command from the command line (w/o opening the python interpreter)
	\end{itemize}

}


\frame {
\frametitle{Measuring the size of presidents' vocabularies }

	Write a bash script which accepts text from stdin, and then uses shell commands and variables as well as python commands to:
	
	\begin{itemize}
		\item remove punctuation (,.;:) from the text
		\item output the total number of \textbf{types} (unique tokens) in the text
		\item output the total number of \textbf{tokens} in the text
		\item output \underline{as a float} the \textbf{lexical diversity} (types/tokens) of the text 
	\end{itemize}

	Then, use your script to find the types, tokens, and lexical diversity of three U.S. presidential inaugural addresses of your choice from:

	\ \\ 

	\url{/usr/share/nltk_data/corpora/inaugural/}

}


\frame {
\frametitle{Measuring the size of presidents' vocabularies }

	What does your numerical value of lexical diversity mean?
	
	\ \\ 
	
	What are some ways you might improve on preprocessing the text before calculating lexical diversity?
	Just think about what you would do, even if you don't know how to do it yet.

}

\end{document}
